\documentclass[french]{article}
\usepackage[T1]{fontenc}
\usepackage{babel, fontspec, lmodern, etoc, hyperref}
\renewcommand{\familydefault}{\sfdefault}
\setlength{\parindent}{0mm}
\setcounter{tocdepth}{2}
\hypersetup{hidelinks}

% Document begins here
\begin{document}
	
	\pagenumbering{gobble}
	
	{\Huge
		La Bouée Corsaire
	}\\ [0.5cm]
	
	% Presentation
	\paragraph{}
		La Bouée Corsaire est un projet de bourse d’échange de services,
		 valorisant les échanges par un système de décompte d’heures pour se
		 passer totalement d’échanges financiers. Ce projet se présente sous la
		 forme d’une application Web où les personnes inscrites peuvent
		 enregistrer des besoins et des services. Les besoins correspondent aux
		 tâches qu’on souhaite voir réalisées mais pour lesquelles on ne pense pas
		 avoir les compétences ou la motivation, et les services sont des tâches
		 qu’on aime effectuer et qu’on se propose de réaliser pour d’autres
		 personnes. Chaque utilisateur du site a un crédit d’heures qui varie de
		 la façon suivante : lorsque un utilisateur répond au besoin d’un autre,
		 la réserve d’heures du premier est créditée du temps passé à résoudre ce
		 besoin, et celle de celui qui avait fait la demande d’aide est débitée du
		 même nombre d’heures. Pour profiter de l’assistance des utilisateurs du
		 projet il faut donc soi-même proposer ses services.
	\paragraph{}
		Cette application doit pouvoir être utilisée au quotidien indépendamment
		 du type de machine cliente utilisée, une emphase particulière est donc
		 mise sur l’accessibilité et la disponibilité.
	\paragraph{}
		L’objectif à long terme du projet est de proposer ce service à des
		 artisans ou PME, ce qui ajoute une contrainte sur les capacités de mise à
		 l’échelle de l’application qui devra pouvoir gérer un panel très large
		 d’utilisateurs.
	
% Document ends here
\end{document}
